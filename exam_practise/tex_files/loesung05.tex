\documentclass{pi1}

\usepackage{listings}
\usepackage{hyperref}

\begin{document}

\maketitle{5}{ Praktikum 13: Mi, 17-20, MZH 0230, Marina Koksharova/Laura Eiermanns }{Ramtin Behesht Aeen}

\section{Zugriff auf Zellen}
\label{sec:aufgabe1}

Ich speichere das Spielfeld in der Klasse \texttt{Field} als \texttt{String}-Array. Ich kann einzelne Zellen lesen.  
Dazu schreibe ich die Methode \texttt{getChar(int x, int y)}, die das Zeichen an der Stelle $(x, y)$ zurückgibt.  
Wenn die Stelle außerhalb ist, gebe ich ein Leerzeichen zurück.
\lstinputlisting[
 language  = Java,
 label     = {lst:field_constructor},
 firstline = 7,
 lastline  = 23
]{Field.java}
\textbf{Erklärung:} Ich speichere das Spielfeld-Array im
 Konstruktor. In \texttt{getChar} prüfe ich, ob
  die Koordinaten gültig sind. Wenn ja, gebe ich das
   Zeichen zurück, sonst ein Leerzeichen.
Ich habe auch die Methode \texttt{getCell(int x, int y)} gemacht. Sie macht das gleiche, gibt das Zeichen an der Stelle $(x, y)$ zurück oder ein Leerzeichen, wenn es außerhalb ist.
\lstinputlisting[
 language  = Java,
 label     = {lst:getcell_full},
 firstline = 47,
 lastline  = 63
]{Field.java}

\textbf{Erklärung:} Ich prüfe zuerst, ob die
 Koordinaten gültig sind. Wenn ja, gebe ich das 
 Zeichen zurück. Sonst gebe ich ein Leerzeichen.
  Der Konstruktor speichert weiter das Spielfeld.

\section{Nachbarschaft berechnen}
\label{sec:aufgabe2}

Ich berechne für
 jede Zelle die Nachbarschaft.
 Die Methode

\begin{lstlisting}[language=Java]
int getNeighborhood(int x, int y)
\end{lstlisting}

gibt eine Zahl zwischen 0 und 15. Die Zahl zeigt, 
welche Nachbarn da sind: rechts (1), unten (2), links (4), oben (8).

\lstinputlisting[
 language  = Java,
 label     = {lst:getneighborhood_full},
 firstline = 63,
 lastline  = 92
]{Field.java}

\section{Spielfeld aufbauen}
\label{sec:aufgabe3}

Im Konstruktor gehe ich durch das Spielfeld. Ich berechne die Nachbarschaft und erstelle GameObjects. Ich benutze zwei \texttt{for}-Schleifen:  

\begin{itemize}
\item Äußere Schleife: geht über jede zweite Zeile (\texttt{y += 2})
\item Innere Schleife: geht über jede zweite Spalte (\texttt{x += 2})
\end{itemize}

\lstinputlisting[
 language  = Java,
 label     = {lst:testfield_full},
 firstline = 93,
 lastline  = 123
]{Field.java}

\textbf{Erklärung:} Für jede Zelle rufe ich \texttt{getNeighborhood} auf. Ich nehme 
die Zahl als Index in \texttt{neighborhoodToFilename}.
 Dann mache ich ein neues GameObject an der halbierten Position.

\textbf{Test:} Ich erstelle ein Testfeld 
und mache für alle Zellen GameObjects. So sehe ich, dass \texttt{getCell} und \texttt{getNeighborhood} funktionieren.

\bibliography{Referenzen}

\end{document}

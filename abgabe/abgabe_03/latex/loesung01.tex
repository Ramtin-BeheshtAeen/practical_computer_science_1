\documentclass{pi1}

\begin{document}

% Titelzeile
\maketitle{3}{ Praktikum 13: Mi, 17-20, MZH 0230, Marina Koksharova/Laura Eiermanns }{Ramtin}

\section{Aufgabe 1 Einmal mit Klasse}
\label{sec:aufgabe1}
\textbf{Der folgende Text und Code wurden von mir erstellt.
Da ich kein Muttersprachler bin, habe ich den Text in ChatGPT
eingefügt, um Wörter und Sätze zu korrigieren. ChatGPT hat den 
Inhalt inhaltlich nicht verändert, sondern nur sprachlich verbessert.
Ob der Text nun etwas ‚roboterhaft‘ klingt, kann ich nicht genau
beurteilen, aber die Anzahl der grammatischen und sprachlichen Fehler 
ist deutlich geringer}

In dieser Aufgabe wurde eine neue Klasse \texttt{NPC} erstellt, 
die ein nicht steuerbares Spielobjekt beschreibt. 
Sie enthält Attribute für die Figur, die maximale Anzahl der Schritte 
und den aktuellen Schritt.

\lstinputlisting[
language=Java,
firstline=0,
lastline=10
]{NPC.java}

\textbf{Beschreibung:}  
Diese Attribute speichern, welche Figur der NPC steuert 
und wie weit sie sich auf ihrem Weg bewegen darf.


\section{Aufgabe 2 Konstruktivismus}
\label{sec:aufgabe2}

Es wurde ein Konstruktor definiert, der die drei Attribute initialisiert.

\lstinputlisting[
language=Java,
firstline=12,
lastline=17
]{NPC.java}

\textbf{Beschreibung:}  
Der Konstruktor weist jedem NPC sein Spielfigur-Objekt, 
die maximale Schrittlänge und die aktuelle Position zu.


\section{Aufgabe 3 Schritt für Schritt}
\label{sec:aufgabe3}

Die Klasse wurde um eine Methode \texttt{act()} erweitert, 
die einen einzelnen Bewegungsschritt ausführt. 
Die Figur läuft einen Schritt vorwärts, zählt diesen, 
und wenn die maximale Schrittzahl erreicht ist, 
dreht sie sich um und setzt den Zähler zurück.

\lstinputlisting[
language=Java,
firstline=19,
lastline=41
]{NPC.java}

\textbf{Beschreibung:}  
Der NPC bewegt seine Figur abwechselnd hin und her, 
indem er die Rotation um 2 erhöht, sobald die maximale Schrittlänge erreicht wurde.


\section{Aufgabe 4 Und Action!}
\label{sec:aufgabe4}

In der Hauptmethode des Spiels (\texttt{NPCGame}) wurden mehrere NPCs erstellt, 
und deren \texttt{act()}-Methoden werden nach jeder Spielerbewegung aufgerufen.

\lstinputlisting[
language=Java,
firstline=46,
lastline=97
]{NPCGame.java}

\textbf{Beschreibung:}  
Nach jedem Tastendruck bewegt sich der Spieler, 
danach führen die NPCs einen Schritt aus. 
Die Figuren reagieren also unabhängig vom Spieler.



\section{Aufgabe 5 Bonusaufgabe}
\label{sec:aufgabe5}

Die NPC-Objekte überprüfen in ihrer Methode \texttt{checkCollision()}, 
ob sie die gleiche Position wie der Spieler erreicht haben. 
Falls ja, wird die Spielfigur unsichtbar und das Spiel endet.

\lstinputlisting[
language=Java,
firstline=43,
lastline=51
]{NPC.java}

\textbf{Beschreibung:}  
Sobald der NPC die gleiche Position wie der Spieler hat, 
wird der Spieler mit \texttt{setVisible(false)} entfernt 
und die Hauptschleife beendet sich automatisch.


\end{document}
